% !Mode:: "TeX:UTF-8"

%%%%%%%%%% Package %%%%%%%%%%%%
\usepackage{CJK}
\usepackage{lmodern}
\usepackage[T1]{fontenc}
\usepackage{graphicx}                       % 支持插图处理
\usepackage[a4paper, top= 35 true mm,left=25 true mm, bottom = 25 true mm, right=25 true mm, head=6true mm,headsep=6.2true mm,foot=16.5true mm]{geometry}
                                            % 支持版面尺寸设置
\usepackage[squaren]{SIunits}               % 支持国际标准单位

\usepackage{titlesec}                       % 控制标题的宏包
\usepackage{titletoc}                       % 控制目录的宏包
\usepackage{fancyhdr}                       % fancyhdr宏包 支持页眉和页脚的相关定义
%\usepackage{ctex}                     % 支持中文显示
\usepackage{CJKpunct}                       % 精细调整中文的标点符号
\usepackage{color}                          % 支持彩色
\usepackage{amsmath}                        % AMSLaTeX宏包 用来排出更加漂亮的公式
\usepackage{amssymb}                        % 数学符号生成命令
\usepackage[below]{placeins}    %允许上一个section的浮动图形出现在下一个section的开始部分,还提供\FloatBarrier命令,使所有未处理的浮动图形立即被处理
\usepackage{multirow}                       % 使用Multirow宏包,使得表格可以合并多个row格
\usepackage{booktabs}                       % 表格,横的粗线;\specialrule{1pt}{0pt}{0pt}
\usepackage{longtable}                      % 支持跨页的表格。
\usepackage{tabularx}                       % 自动设置表格的列宽
\usepackage{subfigure}                      % 支持子图 %centerlast 设置最后一行是否居中
\usepackage[subfigure]{ccaption}            % 支持子图的中文标题
\usepackage[sort&compress,numbers]{natbib}  % 支持引用缩写的宏包
\usepackage{enumitem}                       % 使用enumitem宏包,改变列表项的格式
\usepackage{calc}                           % 长度可以用+ - * / 进行计算
\usepackage{txfonts}                        % 字体宏包
\usepackage{bm}                             % 处理数学公式中的黑斜体的宏包
\usepackage[amsmath,thmmarks,hyperref]{ntheorem}  % 定理类环境宏包,其中 amsmath 选项用来兼容 AMS LaTeX 的宏包
\usepackage{CJKnumb}                        % 提供将阿拉伯数字转换成中文数字的命令
\usepackage{indentfirst}                    % 首行缩进宏包
\usepackage{CJKutf8}                        % 用在UTF8编码环境下,它可以自动调用CJK,同时针对UTF8编码作了设置
%\usepackage{hypbmsec}                      % 用来控制书签中标题显示内容
\newcommand{\tabincell}[2]{\begin{tabular}{@{}#1@{}}#2\end{tabular}}
\usepackage{xcolor}
%支持代码环境
\usepackage{listings}
\lstset{numbers=left,
language=[ANSI]{C},
numberstyle=\tiny,
extendedchars=false,
showstringspaces=false,
breakatwhitespace=false,
breaklines=true,
captionpos=b,
keywordstyle=\color{blue!70},
commentstyle=\color{red!50!green!50!blue!50},
frame=shadowbox,
rulesepcolor=\color{red!20!green!20!blue!20}
}
%支持算法环境
\usepackage[boxed,ruled,lined]{algorithm2e}
\usepackage{algorithmic}

\usepackage{array}
\newcommand{\PreserveBackslash}[1]{\let\temp=\\#1\let\\=\temp}
\newcolumntype{C}[1]{>{\PreserveBackslash\centering}p{#1}}
\newcolumntype{R}[1]{>{\PreserveBackslash\raggedleft}p{#1}}
\newcolumntype{L}[1]{>{\PreserveBackslash\raggedright}p{#1}}

\def\atemp{xelatex}\ifx\atemp\usewhat
\usepackage[unicode,
            pdfstartview=FitH,
            bookmarksnumbered=true,
            bookmarksopen=true,
            colorlinks=false,
            pdfborder={0 0 1},
            citecolor=blue,
            linkcolor=red,
            anchorcolor=green,
            urlcolor=blue,
            breaklinks=true
            ]{hyperref}
\fi

% !Mode:: "TeX:UTF-8"
% !TEX root = tjumain.tex

\iffalse
\bibliography{reference/reference.bib} % 欺骗latextools获取bib文件
\fi

%%%%%%% 正文 %%%%%%%

\setcounter{page}{1}                                 % 单独从 1 开始编页码
\song \xiaosionetwo
\pagenumbering{arabic}

\section{引言}

生态环境以及生态文明建设是中国特色社会主义建设的重要组成部分。自2012年中共十八大首次将生态文明建设纳入国家发展战略以来,我国对生态文明的重视程度显著提升。尤其是在十九大中,通过了成立生态环境部的决议,进一步加强了生态环境治理,实施了生态环境损害责任追究终身制。生态文明的建设、生态环境的保护已经成为了我国进一步深化改革的重要目标。一方面,考虑到人类命运共同体对于生态环境的朴素共识以及代际公平,生态发展是高质量可持续发展体系的重要组成部分。另一方面,狭义上,地球的生态环境是人类活动的载体,经济活动的开展必须要建立在生态安全的基础上。故而,对于生态安全给出适时的测度并探究同经济系统协调关系的现状及外来趋势十分重要。

\subsection{研究背景}
2023年,国际劳工组织发布的《世界就业和社会展望:2023年趋势报告》指出,15岁至24岁的青年在寻找和保持体面就业上面临严重困难,这一群体的失业率是25岁及以上成年人的3倍。2024年3月,16岁~24岁青年人失业率高达19.6%,包括大学生在内的青年就业,仍然面临困难和挑战。就业是民生之本,青年就业问题更是重中之重,稳定和扩大青年人就业需要持续加力。青年人在人力资本方面通常比24岁以上的成人群体拥有较低或不足的人力资本,尤其在特殊技能方面存在明显差距。初次进入劳动力市场的青年缺乏求职、职业规划和发展经验,面临就业困难。学校教育与市场需求不匹配加剧了这一问题。经济衰退时,青年员工往往首当其冲失业,而经济复苏时企业更倾向于招聘有经验的员工,延缓青年就业机会。信息完全、职业搜寻成本高这些因素共同导致青年失业率居高不下,阻碍着他们顺利进入职场。

\newpage

\section{特征事实}

查询国家数据得到我国 2018 年 Q1 到 2023 年 Q1 的全国城镇登记失业率全国 16-24 岁人口城镇调查失业率和实际 GDP 偏离趋势的百分比如下表\ref{select_variable}。

% Please add the following required packages to your document preamble:
% \usepackage{booktabs}
% \usepackage{graphicx}
\begin{table}[htbp]
\centering
\caption{全国城镇调查失业率、全国城镇16-24岁劳动力失业率、实际GDP偏离率数据图}
\label{select_variable}
\resizebox{\textwidth}{!}{%
\begin{tabular}{@{}cccc@{}}
\toprule
       & 全国城镇调查失业率(\%) & 全国城镇16-24岁劳动力失业率(\%) & 实际GDP偏离百分比persent \\ \midrule
2018q1 & 5.033333333   & 10.86666667          & -9.72194          \\
2018q2 & 4.833333333   & 9.9                  & 7.78481           \\
2018q3 & 5             & 12.53333333          & 0.61241           \\
2018q4 & 4.866666667   & 9.966666667          & -3.6446           \\
2019q1 & 5.2           & 11.16666667          & -8.7514           \\
2019q2 & 5.033333333   & 10.66666667          & 8.72291           \\
2019q3 & 5.233333333   & 13.33333333          & 2.117             \\
2019q4 & 5.133333333   & 12.36666667          & -3.35855          \\
2020q1 & 5.8           & 13.13333333          & -6.45334          \\
2020q2 & 5.866666667   & 14.66666667          & 12.26371          \\
2020q3 & 5.566666667   & 16.2                 & 5.13827           \\
2020q4 & 5.233333333   & 12.76666667          & 3.42567           \\
2021q1 & 5.4           & 13.13333333          & -3.5834           \\
2021q2 & 5.033333333   & 14.26666667          & 5.38749           \\
2021q3 & 5.033333333   & 15.36666667          & -2.34683          \\
2021q4 & 5             & 14.26666667          & -6.43326          \\
2022q1 & 5.533333333   & 15.53333333          & -27.93995         \\
2022q2 & 5.833333333   & 18.63333333          & 7.33902           \\
2022q3 & 5.4           & 18.83333333          & 1.36182           \\
2022q4 & 5.566666667   & 17.23333333          & -0.70814          \\
2023q1 & 5.466666667   & 18.33333333          & -9.19902          \\
2023q2 & 5.2           & 20.83333333          & 10.25975          \\
2023q3 & 5.166666667   &                      & 4.57155           \\
2023q4 & 5.033333333   &                      & 2.65827           \\
2024q1 & 5.233333333   &                      & -5.61693          \\ \bottomrule
\end{tabular}%
}
\end{table}
% Please add the following required packages to your document preamble:
% \usepackage{multirow}

由图表可以看出,自2018第一季度至2024第一季度,全国城镇16-24岁劳动力失业率稳定提高,这说明越来越多的青年、大学生面临着失业的困境。

国家统计局副局长盛来运在国新办新闻发布会上表示,今年新进入劳动力市场的大学毕业生是1176万人,尽管是各级党委政府非常重视,也在积极地为大学生就业创造有利的条件。但从一季度情况来看,青年失业率还是略有上升的,需要高度关注。

青年失业率呈现季节性失业与长期趋势叠加态势,摩擦性失业转向长期失业。失业导致人力资本损失,加剧结构性就业矛盾。大学生失业困难持续,失业率难恢复。需深入了解长期失业情况,关注个体失业时间统计。针对不同失业阶段的大学毕业生制定差异化政策,这对就业政策体系提出了新的挑战。

\section{DMP模型的概念与提出}

根据传统理论,劳动市场理应自行运作,让求职者找到职缺。但事实并非如此,原因是在求职过程中,劳资双方均需付出时间及资源寻找适合对象;即使双方“情投意合”,亦可能因工资谈不拢而无法雇用,导致劳动市场上一方面有大量职位空缺,另一方面却有很多人失业。

来自美国麻省理工学院的戴蒙德(Diamond),于1971年首次就上述经济现象解释提出基础理论。他发现,即使是些微的搜寻成本,都会造成与传统“竞争平衡”模式完全不同的配对结果。莫滕森(Mortensen)和皮萨里季斯(Pissarides)进一步拓展这个搜寻理论,并应用到劳动市场,提出著名的“DMP模型”。

\subsection{搜寻和搜寻理论的缘起}

“搜寻(search)”最早由斯蒂格勒(1982年诺奖得主)于1961年提出,他对“搜寻”的定义是:买者在购买前总要询问许多卖者以确定最合适的价格,这种现象称为“搜寻信息不对称”。

价格离散(price dispersion)是搜寻的前提。具有相同质量的商品常常以不同的价格出售,并且都有人购买。这种不同价格的序列,称为价格离散。

搜寻理论的核心观点是:信息搜寻是有成本的。随着搜寻次数的增加,搜寻的边际收益将下降的。当搜寻的预期边际收益等于边际成本时,搜寻活动将会停止。

\subsection{基于DMP模型下的定性分析}

从DMP的框架下进行分析可知,市场上本存在着大量的工人与职位按一定的速率随时间的消逝而流失,而工人也按一定的速率找到工作。而对于大学生而言,“贴合专业”“工薪较高”“体面”等是其客观上具有的择业标准,在此种标准下,搜寻合适职位的时间会被拉长,相应的,市场上原本存在的职位也随之减少,从而使其难以进去一段工作,陷入失业的境遇中。而党经济衰退或是经济复苏时,企业更加偏向裁掉那些入职时长较短,工作经验较少的人群;经济复苏时,又偏向招聘那些工作经验较多,专业能力较强的人群。因此,青少年,抑或是大学生群体极难在经济衰退或复苏中找到一份较为合适的工作,从而陷入失业困境。
若是以外部政策干预,强制增加员工失业补贴,那么在面临裁员时,企业将考虑裁员的收益是否高于该员工未来会带来的预期收益,从而相对少的进行裁员。同时,员工的搜寻欲望也会上升,从而带来就业率的上升。

\section{结论以及政策改善}

在学历贬值的当下,大学生群体作为工作经验较弱的群体极易遭受到经济变动带来的影响,因此,针对此类群体,国家应该推出一系列保障措施,针对大学生的就业开展补贴政策;而对于学校,则应加强其专业素养的培育以及对接社会的教育,从而使其能更好的适应社会就业以及未来发展。



%%%%%%% 结论 %%%%%%%


